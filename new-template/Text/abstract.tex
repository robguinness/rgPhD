This thesis examines the topic of context awareness for navigation applications. Context awareness can be defined as a computer's ability to understand the situation or context in which it is operating. In particular, we are interested in how context awareness can be used to understand the navigation needs of people using mobile computers, such as smartphones, but context awareness can also benefit other types of navigation users, such as maritime navigators. There are countless other potential applications of context awareness, but this thesis focuses on applications related to navigation. For example, if a smartphone-based navigation system can understand when a user is walking, driving a car, or riding a train, then it can adapt its navigation algorithms to improve positioning performance.

We argue that the primary set of tools available for generating context awareness is machine learning. Machine learning is, in fact, a collection of many different algorithms and techniques for developing ``computer systems that automatically improve their performance through experience'' \cite{Mitchell1990}. This thesis examines systematically the ability of existing algorithms from machine learning to endow computing systems with context awareness. Specifically, we apply machine learning techniques to tackle three different tasks related to context awareness and having applications in the field of navigation:  (1) to recognize the activity of a smartphone user in an indoor office environment, (2) to recognize the mode of motion that a smartphone user is undergoing outdoors, (3) to determine the optimal path of a ship traveling through ice-covered waters. The diversity of these tasks was chosen intentionally to demonstrate the breadth of problems encompassed by the topic of context awareness.

During the course of studying context awareness, we adopted two conceptual ``frameworks,'' which we find useful for the purpose of solidifying the abstract concepts of context and context awareness. The first such framework is based strongly on the writings of a rhetorician from Hellenistic Greece, Hermagoras of Temnos, who defined seven elements of ``circumstance''. We adopt these seven elements to describe contextual information. The second framework, which we dub the ``context pyramid'' describes the processing of raw sensor data into contextual information in terms of six different levels. At the top of the pyramid is ``rich context'', where the information is expressed in prose, and the goal for the computer is to mimic the way that a human would describe a situation.

We are still a long way off from computers being able to match a human's ability to understand and describe context, but this thesis improves the state-of-the-art in context awareness for navigation applications. For some particular tasks, machine learning has succeeded in outperforming humans, and in the future there are likely to be tasks in navigation where computers outperform humans. One example might be the route optimization task described above. This is an example of a task where many different types of information must be fused in non-obvious ways, and it may be that computer algorithms can find better routes through ice-covered waters than even well-trained human navigators. This thesis provides only preliminary evidence of this possibility, and future work is needed to further develop the techniques outlined here. The same can be said of the other two navigation-related tasks examined in this thesis.