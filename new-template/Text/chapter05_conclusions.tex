\chapter{Conclusions}
\label{ch:conclusions}

This chapter offers some conclusions based on this thesis, together with the included publications. It is organized as follows: Section~\ref{sec:summary} briefly summarizes the thesis. Section~\ref{sec:main_findings} outlines our main findings. Section~\ref{sec:significance} explains the significance or potential impact of the results. Section~\ref{sec:future_work} describes our future work planned in the areas addressed by this thesis. Finally, Section~\ref{sec:concluding_remarks} provides a few concluding remarks.

\section{Summary}
\label{sec:summary}

The overall goal of this thesis was to improve our understanding of how computing devices can better understand us and our needs. As argued in this thesis, such understanding is often embodied, at least partly, in a concept known as context awareness. The primary method used to endow computers with context awareness has been---and we argue it will continue to be---machine learning. 

In examining these topics, we have narrowed the focus to application areas related to navigation. Despite this narrowing of application areas, there are still many diverse needs in navigation, and this thesis focused on three particular use cases within navigation where context awareness is deemed beneficial: (1) detecting of different human activities inside a typical office environment to improve indoor location tracking, (2) detecting different ``mobility contexts'' of a smartphone user to improve outdoor location tracking, and (3) enabling ``ice aware'' route optimization for ships sailing in ice-covered waters to improve and automate the route planning needs of such ships. These use cases demonstrate the breadth of potential application areas of context aware technology. Three of the included publications ([P3]--[P5]) aim to improve the state-of-the-art in these application areas by introducing either novel methods, novel combinations of existing methods, or in-depth analysis of the performance of existing methods.

In addition to examining these application areas, this thesis has extensively reviewed the literature concerning context awareness and machine learning. In presenting and summarizing these topics, we have attempted to provide clear, tutorial-like examples, in order to aid readers unfamiliar with these subjects.

We have reviewed the early theoretical work in ``context,'' led by artificial intelligence pioneer John McCarthy and others, while pointing out that generalizations of context have not led to significant breakthroughs in context-aware systems. In presenting the conceptual underpinnings of context awareness, we have introduced two conceptual frameworks for understanding context awareness and contextual reasoning. The first was adapted from the writings of an ancient Greek orator seven Hermagoras, known as the ``seven circumstances''. The second, which we have dubbed the ``context pyramid'' presents a division of the various steps in contextual reasoning into six levels ranging from raw data to ``rich context''. These two frameworks, general in nature, can assist the researcher and developer aiming to build context-aware systems by dividing the problem up into different categories of contextual information and steps in contextual reasoning.

On the topic of machine learning, this thesis has examined the original goal of machine learning, as envisioned by pioneers such as Arthur Samuel. In the tradition of Samuel and Shannon, we elucidate the concept of \emph{automatic learning} using the toy example of computer chess. We then examine the modern notion of machine learning, including the two major types, supervised learning and unsupervised learning. We provide a tutorial-like example of both types of learning, using an example problem from context awareness. In particular, we have emphasized the importance and benefits of automatic learning. That is, supervised learning usually requires manual labeling of training data, whereas unsupervised learning can largely meet the desire for automated learning, although it often requires some human interpretation of the results. 

Finally, the included publications provide further details on machine learning its application to context awareness, and in particular [P3] and [P4] demonstrate the use of machine learning in practice. Lastly, [P5] provides an example application of context awareness in the field of navigation, i.e. an ice-aware route optimization method.

\section{Main Findings}
\label{sec:main_findings}

Below summarizes our main research findings:
%
\begin{itemize}
\item Many aspects of context awareness are feasible to realize using only the sensors in smartphones. For example, we have shown that smartphones can reliably detect different mobility contexts (\textgreater97\% recall rate) and detect different office-environment activities (\textgreater90\% recall rate). We have also introduced a method to detect whether a user is indoors or outdoors. Existing algorithms from machine learning, especially supervised learning, provide adequate levels of performance for these use cases, although generalization to large user populations will require the collection of more extensive training data. Nonetheless, context-aware smartphone applications are presently feasible and can be realized using existing machine learning techniques.
%
\item Although machine learning constitutes a powerful set of methods for endowing computers with context awareness, a systematic evaluation of different available machine learning algorithms should be undertaken when applying machine learning to the problem of context awareness, especially if the aim is to maximize performance. The important fact is often overlooked by navigation researchers working on context awareness. After evaluating the performance of 20 different supervised learning algorithms, we determined that the RandomForest algorithm performed the best on our dataset. Our results also showed that performance is optimized only after applying extensive feature selection and parameter tuning. Another general recommendation from our research is that, due to the need to train many different classification models (with various feature sets and parameter settings), a high level of automation for this type of analysis is desirable, and we developed some software tools to improve the automation of this type of analysis.
%
\item As an example of a maritime application, awareness about ice conditions (as a function of space and time) can be exploited to perform automated route optimization. Such capability could augment or even replace the currently human-intensive task of route planning performed by crews of ships sailing in ice covered waters. Our research showed that graph-based approaches are feasible for modeling the maritime transportation in ice-covered waters and that the A* algorithm can be applied to find optimal paths. In order to realize this implementation of the A* algorithm, our research presents a simple but novel cost function that takes into account the operational constraints posed by ice breaker assistance. Essentially, this cost function captures contextual information about a ship's theoretical speed through an ice field, taking into account the ship's own ice-breaking performance and possible assistance from an ice breaker.
%
\item Our research suggests not only the feasibility of context-aware applications related to the three use case scenarios investigated, but also that many other applications of context awareness are evident in emerging technologies. We believe context awareness will play an even stronger role in the future, especially in so-called ``smart devices'' such as smartphones and smartwatches.
%
\end{itemize}

\section{Significance of the Results}
\label{sec:significance}

This thesis contributes to the overall body of knowledge on context awareness, which as previously discussed, has many potential applications. We have introduced two general and flexible frameworks related to context awareness---one for the systematic encoding of contextual information and the other for the processing of raw sensor data into ``rich context''. These frameworks serve as a methodological skeleton on which other researchers and developers can build new context-aware systems.

Much of this thesis focuses on context awareness that can be achieved using only the sensors in smartphones. Because of the widespread prevalence of smartphones in modern society, the results of smartphone-based context awareness have a strong potential for widespread adoption. We can compare this to previous context awareness studies, for example those related to office environments, that rely on custom sensors being installed in the environment; such studies suffer from the limitation that they require installation of new hardware into the environment, increasing the cost of adoption. According to our knowledge, our research is the first to look at office-related context awareness utilizing only smartphone sensors and standard WLAN access points.

Regarding the performance results reported in our mobility context research, it is difficult to determine definitively whether or not our results represent an improvement over the state-of-the-art. This is due to the fact that we can not compare our results against those obtained with comparable datasets. This issue will be further discussed below in Section~\ref{sec:general_issues}. What we can conclude confidently is that extensive analysis, consisting of evaluating different machine learning algorithms, performing feature selection, and systematically tuning parameters of the algorithm, will result in better performance regardless of the dataset. Surprisingly, many research works in context awareness, especially in the navigation community, overlook this fact and report results from evaluating only one or a few machine learning algorithms and without reporting any results of parameter tuning. We hope that the methodology described in this thesis informs other researchers as to the benefits of such detailed analysis.

Lastly, our research demonstrates the feasibility of developing an ice aware maritime navigation system, specifically to provide automatic route optimization. Although further refinement is needed, the proposed methods have the potential to provide significant savings for the maritime transportation system. After presenting these methods to Director-General of the Finnish Ministry of Transportation and Communications, Mr. Pekka Plathan commented that this research has the potential to save billions for the Finnish maritime industry.

Ice-aware route optimization can not only bring economic benefits for the maritime industry, but it also has the potential to provide safety benefits. The methods described in this thesis are flexible in that the cost function can be modified to minimize non-economic factors as well. For example, using models for the risk that ice poses towards damaging ships or getting the ship completely stuck in the ice, one can design an appropriate cost function aimed to minimize these risks.

\section{Future Work}
\label{sec:future_work}

In many ways, this thesis has only scratched the surface in exploring context awareness for navigation applications. In tackling the broader goal ``to improve our understanding of how computing devices can better understand us and our needs,'' we feel even less compelled to declare our work complete. This section outlines some of the planned future work in developing context-aware navigation applications.

Our future work can be divided into three broad categories: (1) future work in the three application areas covered by the included publications, (2) future work in new application areas, and (3) future work that can benefit context awareness broadly. These areas of future work will be discussed in separate sub-sections below.
%
\subsection{Future Work in Investigated Applications}
\label{sec:future_applications}

In our research on detecting office-environment contexts, we investigated a small number of different workplace contexts, including working in one's office, having lunch, taking a break, and fetching coffee or water. There are obviously a large number of other workplace contexts that could be investigated, such as having a meeting, giving a presentation, having an impromptu conversation, talking on the phone, etc. In our research, we grouped all ``non-defined'' contexts into a single category called ``undefined context''. Also, we conducted this research in only one particular office environment. Extending this research to many diverse office environments and types of work would also improve the robustness of the results.

Another way to improve this line of research would be to expand the sources of raw data to include other sensors. For example, smartphones have microphones and sound could be an important source of contextual information. In fact, we informally explored using audio as a feature, but there are some challenges in this regard. For example, when the phone is in the pocket, the audio signal becomes very muffled and contact with clothing can cause loud undesirable signals. Nonetheless, we believe audio is an important source of contextual information, and we aim to explore this further in the future.

Also, different social context aspects of the workplace environment will be included in the future. In this thesis, we did not address social context at any depth, but especially in an office environment, it should be feasible to recognize different social contexts because in many workplaces the identities of most of the people present is largely known. Using various proximity sensing technologies, a smartphone could apply contextual reasoning about different social contexts, such as ``with the boss,'' ``with subordinates,'' ``with colleagues,'' ``with a customer,'' etc. Such contextual information may not be necessarily needed for navigation applications, but it would certainly have other applications related to mobile computing.

On the subject of mobility contexts, we also plan to expand the range of contexts under investigation, such as cycling, riding trams, riding metros, etc. In addition, we plan to investigate whether we can recognize a number of other mobility-related leisure activities, such as hiking, berry or mushroom picking\footnote{These are popular leisure activities in Finland.}, playing golf, dog-walking, etc.

Also, in this thesis, for detecting mobility context we only utilized two smartphone sensor types, namely GPS and accelerometers. In the future, we aim to include other types of sensors, such as gyroscopes, pressure sensors, microphones, and light sensors.

Lastly, we plan to expand the number of test subjects participating in the collection of training data. This is important to ensure that the trained models are robust. In this thesis, we asked the test subject to keep the smartphone in a particular location (pants pocket), so in future research we will also study the effects of placement of the smartphone in different locations, such as a backpack, handbag, belt ``holster,'' etc.

On the subject of ice-aware route optimization, further work is needed to validate the proposed method. This should include further analysis of historical data from \gls{ais}, as well as simulator-based studies and actual testing of routes at sea. Also, in this thesis the route optimization method focused on minimizing travel times for ships, but in the future other aspects should be investigated, such as fuel usage, operational efficiency, safety, and reliability. Lastly, contextual information regarding the ice conditions should be enriched compared to the model used in this thesis. For example, the current model does not take into account ice compression, which can have a large effect on ice-going ship performance.
%
\subsection{Future Applications}
\label{sec:future_applications}

In Figure~\ref{fig:publication-chart} we hinted at future application areas or ``use case scenarios''. Several of the highlighted use case scenarios are part of near future work. For example, in one recently initiated project, we aim to develop a ``tactical situation awareness system'' for soldiers.

Military applications of context awareness are particularly promising because the cost limitations are not as strict as in other application areas and specially-designed sensors can be installed, e.g. attached to various body parts of a soldier (helmet, boots, chest, etc.) or to other military equipment, providing a rich set of raw sensor data from which to generate context awareness. On the other hand, in military applications, reliability requirements are very high, and typically there is a strong requirement for real-time functionality. For example, if a system is designed to detect when a soldier is in danger or injured, then false negatives, as well as false positives could prove very costly.

Another application area that has strong potential is healthcare and fitness monitoring. With the growing popularity of ``wearable devices,'' such as smartwatches and small heart-rate monitors, such applications have greater widespread consumer appeal. Many devices already exist that can, e.g. monitor calorie usage by tracking steps, but it remains a challenge to reliably and automatically detect different activities such as walking, running, cycling, hiking, etc. This is, of course, strongly overlapping with the topic of [P4], but we believe healthcare and fitness monitoring can go much beyond mere ``mobility context'' and incorporate other aspects, such as recognizing social interactions, detecting abnormal health or changes to a person's routine that might affect health and fitness, and warning users of dangerous or unhealthy situations. The concept of a ``personal health assistant'' is not really a matter of science-fiction but could be realized in the coming years. Context awareness and machine learning are the technologies that are likely to make this concept a reality.

\subsection{General Issues and Potential Solutions}
\label{sec:general_issues}

Lastly, we have noticed in our research several general issues that are relevant to context awareness in a broad sense. These issues are summarized as follows:
%
\begin{enumerate}
 \item Supervised learning requires labeled data, and labeled data is expensive.
 \item There is a lack of standardization in context awareness research.
 \item Many context awareness experiments are not easy to repeat or independently verify.
\end{enumerate}


The first general issue above is related to the use of supervised learning, which is often the adopted approach in many research works (such as in [P3] and [P4]). While supervised learning has many advantages compared to unsupervised learning, it can be very costly and time-consuming to generate the required labeled data. Furthermore, it is generally the case that the more data that can be collected, the more performant and reliable the resulting model will be. For example, if we are aiming to develop a context-aware smartphone application that works well across a large population of users, then we will need to collect training data from a large, diverse population of test users. This is very costly, especially in a research setting.

There are two potential solutions to this issue. The first is that researchers and developers would publish and share their training data. This would benefit the overall research community. We have practiced this approach in publication [P4], and there are a few other examples in the literature (e.g. \cite{yu2014big} \cite{ordonez2013activity}). Generally, this is not a common practice in context awareness research. The second approach would be to collect a sizable amount of labeled training data and then to supplement it with unlabeled data (which is less costly to collect). Performing machine learning using a combination of labeled and unlabeled data is known as \emph{semi-supervised learning}. This topic is outside the scope of this thesis but will be explored in our future work. As an example of this approach, a research and development team could collect a limited amount of labeled data using its own staff and volunteers and then supplement it with a large amount of crowdsourced unlabeled data. This is exactly the approach we are taking in a recently initiated project called MyGeoTrust (see \cite{Guinness2015}).

The second general issue has to do with standardization. To put it precisely, there is a lack of standardization in context awareness research, and this issue makes it difficult to compare results among different studies. As described in Chapter~\ref{ch:context_awareness}, context is understood in many different ways, and there is no one ``correct'' way to categorize and organize the context space. Table~2.1 demonstrates this problem.

This lack of standardization is understandable, due to the fact that different researchers have different applications in mind and different ideas about how to segregate the context space, but it would be more beneficial for the overall research community if some level of standardization were applied. Many context ontologies have been proposed in the literature (e.g. \cite{guermah2014ontology} \cite{chen2003ontology} \cite{wang2004ontology} \cite{gu2004ontology} \cite{strang2003cool}, and one of these could form the basis of a context ontology standard. Then, when presenting results, researchers could reference these standards, i.e. ``the following classification results are according to standard X.Y...''. Also, there is no reason to limit results to one particular standard; data could be processed according to several different standards and presented in the same publication. The problem is that no forerunners for a standard have emerged and no good software tools for working with the proposed ontologies have been made available. There are many possible reasons for this, and we would not like to speculate too deeply in this regard. It suffices to say that in our future work we aim to contribute to and advance the notion of standard context ontologies, including open source tools for working with such ontologies.
% DCON 
%
% S. Scerri, J. Attard, I. Rivera, M. Valla, and S. HandschuhDcon: Interoperable context representation for pervasive
% environments. In In Proceedings of the Activity Context
% Representation Workshop at AAAI 2012, 2012.

% Korpipää et al., 2004 
% Utilising Context Ontology in Mobile Device Application Personalisation

The last general issue we would like to discuss is somewhat related to the second issue, and the solution is ironically similar to the first solution described above. One of the long-standing tenets of scientific research is reproducibility. Experiments should be described in enough detail so that other researchers can independently verify the results. In the case of context awareness research, this means that an independent researcher should be able repeat another researcher's data collection, apply the same algorithms, and achieve similar, if not identical, results. In reality, there are so many factors related to the environment, devices, and test subjects that collecting comparable data that produces comparable results is not always realistic.

The solution is straight-forward. As an alternative, context researchers should always publish the data upon which their results are based, along with sufficient documentation so that the data is usable by independent researchers. As already stated, this is rarely done in context awareness research. It is, however, a common practice in the machine learning community to test techniques against benchmark data. For example, the \gls{uci} maintains a repository of over 300 datasets that can be used for machine learning research \cite{Lichman2015}. Unfortunately, very few of these datasets relate to context awareness. A few other sources of open data for context awareness research exist, including the Mobile Data Challenge (MDC) Dataset collected as part of the Lausanne Data Collection Campaign \cite{laurila2012mobile} and data from the University of Helsinki's ``Context project'' \cite{Raento2005}. We aim to follow open data practices in our future work and also to actively promote this practice, either by promoting the use of UCI's machine learning repository or by setting up a dedicated portal for context awareness research.

\section{Concluding Remarks}
\label{sec:concluding_remarks}

The remainder of this thesis consists of reprints of the five included publications described earlier. The order of the publications has been chosen to go from the most general to more specific and detailed applications. They can be read, however, in any order, depending on the reader's specific interests.

